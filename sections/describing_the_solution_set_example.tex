\documentclass{subfiles}

\begin{document}

\subsubsection{Примерно Уравнение 1}

\begin{equation*}
    \sysdelim..\systeme{
        2x + z = 3@p_{*},
        x - y - z = 1,
        3x - y = 4
    }
    \quad
    \overset{-(\frac{1}{2})p_{1} + p_{2}}{\underset{-(\frac{3}{2})p_{1} + p_{3}}{\longrightarrow}}
    \quad
    \sysdelim..\systeme{
        2x + z = 3,
        -y - \frac{3}{2}z = -\frac{1}{2},
        -y - \frac{3}{2}z = -\frac{1}{2}
    }
    \quad
    \overset{-p_{2}+p_{3}}{\longrightarrow}
    \quad
    \sysdelim..\systeme{
        2x + z = 3,
        -y - \frac{3}{2}z = -\frac{1}{2},
        0 = 0
    }
\end{equation*}

\noindent Така разбираме, че едното от уравненията не е определящо за множеството от решения. Множеството на решенията може да бъде описано като:

\begin{equation*}
    \{(x,y,z)\ |\ 2x + z = 3 \text{ and } -y - \frac{3z}{2} = - \frac{1}{2}
\end{equation*}

\noindent Което е по-дборе, защото уравнения са 2 вместо 3, но има и по-добър начин. Използвайки \textbf{свободната променливата} $z$, която \textbf{не води уравнение}, за да опишем променливите, които \textbf{водят уравнение}, $x$ и $y$:

\begin{align*}
    y = \frac{1}{2} - \frac{3}{2}z \\
    x = \frac{3}{2} - \frac{1}{2}z
\end{align*}

\noindent Така можем да опишем решението като:

\begin{equation*}
    \textstyle
    \{ \left(\left(\frac{3}{2} - \frac{1}{2}z\right), \left(\frac{1}{2} - \frac{3}{2}z\right), z \right) \ |\ z \in \mathbb{R}\}
\end{equation*}

\noindent Предимството е, че $z$ може да бъде всяко реално число. Това значително улеснява работата по определяне на елементите в множеството от решения. Например, ако приемем $z = 2$ тогава веднага виждаме, че $(\frac{1}{2}, −\frac{5}{2}, 2)$ е решение.

\subsubsection{Примерно Уравнение 2}

\noindent Една система може да има повече от една \textbf{свободна променлива}:

\begin{equation*}
    \sysdelim..\systeme[x,y,z,w]{
        x + y + z - w = 1@p_{*},
        y - z + w = -1,
        3x + 6z - 6w = 6,
        -y + z - w = 1
    }
    \quad
    \overset{-3p_{1}+p_{3}}{\longrightarrow}
    \quad
    \sysdelim..\systeme[x,y,z,w]{
        x + y + z - w = 1,
        y - z + w = -1,
        -3y + 3z - 3w = 3,
        -y + z - w = 1
    }
    \quad
    \overset{3p_{2} + p_{3}}{\underset{p_{2} + p_{4}}{\longrightarrow}}
    \quad
    \sysdelim..\systeme[x,y,z,w]{
        x + y + z - w = 1,
        y - z + w = -1,
        0 = 0,
        0 = 0
    }
\end{equation*}

\noindent Oставa $x$ и $y$ водещи и $z$ и $w$ свободни. Първо изразяваме водещата променлива $y$ чрез $z$ и $w$:

\begin{equation*}
    y = −1 + z - w
\end{equation*}

\noindent След това изразяваме $x$ чрез $z$ и $w$:

\begin{equation*}
    x = 2 - 2z + 2w
\end{equation*}

\noindent А това дава следното множество от решения:

\begin{equation*}
    \{ ((2 - 2z + 2w), ( -1 + z - w ), z, w) \ |\ z, w \in \mathbb{R}\}
\end{equation*}

\subsubsection{Примерно Уравнение 3}

\noindent Списъкът с водещи променливи може да пропусне някои колони:

\begin{equation*}
    \sysdelim..\systeme[x,y,z,w]{
        2x - 2y = 0@p_{*},
        z + 3w = 2,
        3x - 3y = 0,
        x - y + 2z + 6w = 4
    }
    \quad
    \overset{-(\frac{3}{2})p_{1} + p_{3}}{\underset{-(\frac{1}{2})p_{1} + p_{4}}{\longrightarrow}}
    \quad
    \sysdelim..\systeme[x,y,z,w]{
        2x - 2y = 0,
        z + 3w = 2,
        0 = 0,
        2z + 6w = 4
    }
    \quad
    \overset{-2p_{2}+p_{4}}{\longrightarrow}
    \quad
    \sysdelim..\systeme[x,y,z,w]{
        2x - 2y = 0,
        z + 3w = 2,
        0 = 0,
        0 = 0
    }
\end{equation*}

\noindent Тук $x$ и $z$ са водещите променливи. Свободните променливи са $y$ и $w$ и затова можем да опишем набора от решения като $\{ (y, y, 2 - 3w, w)\ |\ y, w \in \mathbb{R} \}$. Ако вземем списъкът $(1, 1, 2, 0)$ удовлетворява системата, но $(1, 0, 5, 4)$ не, тъй като първата и втората стойност не са равни.

\noindent Променлива, която използваме за описване на семейство от решения, е параметър. Казваме, че решението, зададено в предишния пример, е параметризирано с $y$ и $w$.

\noindent Термините \textbf{параметър} и \textbf{свободна променлива} не означават едно и също нещо. В предишния пример $y$ и $w$ са свободни, тъй като в системата на ешелонната форма те не водят. Те са параметри, защото ги използвахме за описание на набора от решения.

\subsubsection{Нотация са описване на множеството от решения}

Първата нотация, която използваме, е тази за матрица. Бележи се с голяма латинска буква, а елементите в нея се бележат с малка латинска буква:

\begin{equation*}
    A = \left(
            \begin{array}{ ccc }
                1 & 2.2 & 5 \\
                3 & 4 & -7
            \end{array}
        \right)
\end{equation*}
\begin{align*}
    A_{2 \times 3} \text{ we say that the} & \text{ matrix is a 2 by 3 matrix} \\
    a_{2,1} &= 3 \\
    a_{1,2} &= 2.2 \\
    a_{3,3} &= -7 \\
\end{align*}

\noindent Можем да използваме матричния запис при решаване на уравнение чрез метода на Гаус.

\noindent Например, следната система може да се изрази с матрична нотация:

\begin{equation*}
    \sysdelim..\systeme{
        x + 2y = 4,
        y - z = 0,
        x + 2z = 4
    }
    \quad
    \longrightarrow
    \quad
    \left(
        \begin{array}{ ccc|c }
            1 & 2 & 0 & 4 \\
            0 & 1 & -1 & 0 \\
            1 & 0 & 2 & 4 \\
        \end{array}
    \right)
\end{equation*}

\noindent Стъпките по метода на Гаус:

\begin{equation*}
    \left(
        \begin{array}{ ccc|c }
            1 & 2 & 0 & 4 \\
            0 & 1 & -1 & 0 \\
            1 & 0 & 2 & 4 \\
        \end{array}
    \right)
    \quad
    \overset{-p_{1}+p_{3}}{\longrightarrow}
    \quad
    \left(
        \begin{array}{ ccc|c }
            1 & 2 & 0 & 4 \\
            0 & 1 & -1 & 0 \\
            0 & -2 & 2 & 0 \\
        \end{array}
    \right)
    \quad
    \overset{2p_{2}+p_{3}}{\longrightarrow}
    \quad
    \left(
        \begin{array}{ ccc|c }
            1 & 2 & 0 & 4 \\
            0 & 1 & -1 & 0 \\
            0 & 0 & 0 & 0 \\
        \end{array}
    \right)
\end{equation*}

\noindent Така множеството решения е $\{(4 − 2z, z, z)\ |\ z \in \mathbb{R} \}$, което може да се разпише чрез векторена нотация:

\begin{equation*}
    \{
        \left(\begin{array}{ c }-4 \\ 0 \\ 0 \end{array}\right) +
        \left(\begin{array}{ c }-2 \\ 1 \\ 1  \end{array}\right) \cdot z
        \ | \ z, w \in \mathbb{R}
    \}
\end{equation*}

\noindent Векторите се бележат с малко по различна нотация:
\begin{equation*}
    \overrightarrow{v} = \left(\begin{array}{ c }1 \\ 3 \\ 7\end{array}\right)
\end{equation*}

\end{document}