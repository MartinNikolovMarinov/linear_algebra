\documentclass{subfiles}

\begin{document}

\subsubsection{Задача 1.21}

\begin{equation*}
    \sysdelim..\systeme{
        x + 3y = 1,
        2x + y = -3,
        2x + 2y = 0
    }
\end{equation*}

\noindent Решение по метода от училище.

\noindent Първо првим това което пише в подточка a:

\begin{align*}
    x &= 1 - 3y \\
    2x + y &= -3 \\
    2x + 2y &= 0
\end{align*}
\begin{align*}
    x &= 1 - 3y \\
    2(1 - 3y) + y &= -3 \\
    2x + 2y &= 0
\end{align*}
\begin{align*}
    x &= 1 - 3y \\
    y &= 1 \\
    2x + 2y &= 0
\end{align*}
\begin{align*}
    x &= -2 \\
    y &= 1 \\
    2x + 2y &= 0
\end{align*}
\begin{align*}
    x &= -2 \\
    y &= 1 \\
    -2 &= 0
\end{align*}

\noindent Виждаме противоречие $-2=0$, което можем да хванем само ако заместим в 3тото уравнение.

\noindent Првим това което пише в подточка b:

\begin{align*}
    x &= 1 - 3y \\
    2x + y &= -3 \\
    2x + 2y &= 0
\end{align*}
\begin{align*}
    x &= 1 - 3y \\
    2x + y &= -3 \\
    2(1 - 3y) + 2y &= 0
\end{align*}
\begin{align*}
    x &= 1 - 3y \\
    2x + y &= -3 \\
    y &= \frac{1}{2}
\end{align*}
\begin{align*}
    x &= 1 - 3y \\
    x &= -7 \\
    y &= \frac{1}{2}
\end{align*}
\begin{align*}
    -7 &= - \frac{1}{2} \\
    x &= -7 \\
    y &= \frac{1}{2}
\end{align*}

\noindent Отново виждаме противоречие $-7 = - \frac{1}{2}$ след изришна проверка.

\end{document}
