\documentclass{subfiles}

\begin{document}

\subsubsection{Задача 3.14}

\begin{equation*}
    \sysdelim..\systeme{
        x + y - 2z = 0@p_{*},
        x - y = -3,
        3x - y - 2z = -6,
        2y - 2z = 3
    }
\end{equation*}

\noindent Решение:
\begin{equation*}
    \left(
        \begin{array}{ ccc|c }
            1 & 1  & -2 &  0 \\
            1 & -1 &  0 & -3 \\
            3 & -1 & -2 & -6 \\
            0 & 2  & -2 &  3 \\
        \end{array}
    \right)
    \quad
    \overset{-p_{1}+p_{2}}{\underset{-3p_{1}+p_{3}}{\longrightarrow}}
    \quad
    \left(
    \begin{array}{ ccc|c }
        1 & 1  & -2 &  0 \\
        0 & -2 &  2 & -3 \\
        0 & -4 &  4 & -6 \\
        0 & 2  & -2 &  3 \\
    \end{array}
    \right)
    \quad
    \overset{-2p_{2}+p_{3}}{\underset{p_{2}+p_{4}}{\longrightarrow}}
    \quad
    \left(
    \begin{array}{ ccc|c }
        1 & 1  & -2 &  0 \\
        0 & -2 &  2 & -3 \\
        0 &  0 &  0 &  0 \\
        0 &  0 &  0 &  0 \\
    \end{array}
    \right)
    \quad
    \longrightarrow
    \quad
    \begin{array}{ c }
        x = -\frac{3}{2} + z,\\
        y = \frac{3}{2} + z, \\
        z = z
    \end{array}
\end{equation*}

\noindent Множеството от решения е:
\begin{equation*}
    \{
        \left(\begin{array}{ c } -(3/2) \\ 3/2 \\ 0 \end{array}\right) +
        \left(\begin{array}{ c } 1 \\ 1 \\ 1 \end{array}\right) z
        \ |\ z \in \mathbb{R}
    \}
\end{equation*}

\noindent Ако конвертираме системата към хомогенна:
\begin{equation*}
    \sysdelim..\systeme{
        x + y - 2z = 0@p_{*},
        x - y = 0,
        3x - y - 2z = 0,
        2y - 2z = 0
    }
\end{equation*}

\noindent Стъпките по метода на Гаус са същите, но множеството от решения е:

\begin{equation*}
    \{
        \left(\begin{array}{ c } 1 \\ 1 \\ 1 \end{array}\right) z
        \ |\ z \in \mathbb{R}
    \}
\end{equation*}

\end{document}