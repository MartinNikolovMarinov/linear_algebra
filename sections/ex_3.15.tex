\documentclass{subfiles}

\begin{document}

\subsubsection{Задача 3.15 a)}

\begin{equation*}
    \sysdelim..\systeme{
        3x + 6y = 18@p_{*},
        x + 2y = 6
    }
\end{equation*}

\noindent Решение:
\begin{equation*}
    \left(
        \begin{array}{ cc|c }
            3 & 6 & 18 \\
            1 & 2 & 6
        \end{array}
    \right)
    \quad
    \overset{-\frac{1}{3}p_{1}+p_{2}}{\longrightarrow}
    \quad
    \left(
        \begin{array}{ cc|c }
            3 & 6 & 18 \\
            0 & 0 & 0
        \end{array}
    \right)
    \quad
    \longrightarrow
    \quad
    \begin{array}{ c }
        x = 6 - 2y \\
        y = y
    \end{array}
\end{equation*}

\noindent Множеството от решения е:
\begin{equation*}
    \{
        \left(\begin{array}{ c } 6 \\ 0 \end{array}\right) +
        \left(\begin{array}{ c } -2 \\ 1 \end{array}\right) y
        \ |\ y \in \mathbb{R}
    \}
\end{equation*}

\noindent Специфичното решение и множеството от решения на асоциираната хомогенната система са:
\begin{equation*}
    \{
        \left(\begin{array}{ c } 6 \\ 0 \end{array}\right)
    \}
    \quad
    \text{and}
    \quad
    \{
        \left(\begin{array}{ c } -2 \\ 1 \end{array}\right) y
        \ |\ y \in \mathbb{R}
    \}
\end{equation*}

\subsubsection{Задача 3.15 b)}

\begin{equation*}
    \sysdelim..\systeme{
        x + y = 1@p_{*},
        x - y = -1
    }
\end{equation*}

\noindent Решение:
\begin{equation*}
    \left(
        \begin{array}{ cc|c }
            1 & 1 & 1 \\
            1 & -1 & -1
        \end{array}
    \right)
    \quad
    \overset{-p_{1}+p_{2}}{\longrightarrow}
    \quad
    \left(
        \begin{array}{ cc|c }
            1 & 1 & 1 \\
            0 & -2 & -2
        \end{array}
    \right)
    \quad
    \longrightarrow
    \quad
    \begin{array}{ c }
        x = 1 \\
        y = 1
    \end{array}
\end{equation*}

\noindent Специфичното решение и множеството от решения на асоциираната хомогенната система са:
\begin{equation*}
    \{
        \left(\begin{array}{ c } 1 \\ 1 \end{array}\right)
    \}
    \quad
    \text{and}
    \quad
    \{
        \left(\begin{array}{ c } 0 \\ 0 \end{array}\right)
    \}
\end{equation*}

\subsubsection{Задача 3.15 c)}

\begin{equation*}
    \sysdelim..\systeme{
        x_{1} + x_{3} = 4@p_{*},
        x_{1} - x_{2} + 2x_{3} = 5,
        4x_{1} - x_{2} + 5x_{3} = 17
    }
\end{equation*}

\noindent Решение:
\begin{equation*}
    \left(
        \begin{array}{ ccc|c }
            1 & 0 & 1 & 4 \\
            1 & -1 & 2 & 5 \\
            4 & -1 & 5 & 17
        \end{array}
    \right)
    \quad
    \overset{-p_{1}+p_{2}}{\underset{-4p_{1}+p_{3}}{\longrightarrow}}
    \quad
    \left(
        \begin{array}{ ccc|c }
            1 & 0 & 1 & 4 \\
            0 & -1 & 1 & 1 \\
            0 & -1 & 1 & 1
        \end{array}
    \right)
    \quad
    \overset{-p_{2}+p_{3}}{\longrightarrow}
    \quad
    \left(
        \begin{array}{ ccc|c }
            1 &  0 & 1 & 4 \\
            0 & -1 & 1 & 1 \\
            0 &  0 & 0 & 0
        \end{array}
    \right)
    \quad
    \longrightarrow
    \quad
    \begin{array}{ c }
        x_{1} = 4 - x_{3} \\
        x_{2} = -1 + x_{3} \\
        x_{3} = x_{3}
    \end{array}
\end{equation*}

\noindent Множеството от решения е:
\begin{equation*}
    \{
        \left(\begin{array}{ c } 4 \\ -1 \\ 0 \end{array}\right) +
        \left(\begin{array}{ c } -1 \\ 1 \\ 1 \end{array}\right) x_{3}
        \ |\ x_{3} \in \mathbb{R}
    \}
\end{equation*}

\noindent Специфичното решение и множеството от решения на асоциираната хомогенната система са:
\begin{equation*}
    \{
        \left(\begin{array}{ c } 4 \\ -1 \\ 0 \end{array}\right)
    \}
    \quad
    \text{and}
    \quad
    \{
        \left(\begin{array}{ c } -1 \\ 1 \\ 1 \end{array}\right) x_{3}
        \ |\ x_{3} \in \mathbb{R}
    \}
\end{equation*}

\subsubsection{Задача 3.15 d)}

\begin{equation*}
    \sysdelim..\systeme{
        2a + b - c = 2@p_{*},
        2a + c = 3,
        a - b = 0
    }
\end{equation*}

\noindent Решение:
\begin{equation*}
    \left(
        \begin{array}{ ccc|c }
            2 &  1 & -1 & 2 \\
            2 &  0 & 1 & 3  \\
            1 & -1 & 0 & 0
        \end{array}
    \right)
    \quad
    \overset{-p_{1}+p_{2}}{\underset{-\frac{1}{2}p_{1}+p_{3}}{\longrightarrow}}
    \quad
    \left(
        \begin{array}{ ccc|c }
            2 &  1 & -1 & 2 \\
            0 & -1 &  2 & 1  \\
            0 & -\frac{3}{2} & \frac{1}{2} & -1
        \end{array}
    \right)
    \quad
    \overset{-\frac{3}{2}p_{2}+p_{3}}{\longrightarrow}
    \quad
    \left(
        \begin{array}{ ccc|c }
            2 &  1 & -1 & 2 \\
            0 & -1 &  2 & 1  \\
            0 &  0 & -\frac{5}{2} & -\frac{5}{2}
        \end{array}
    \right)
    \quad
    \longrightarrow
    \quad
    \begin{array}{ c }
        a = 1 \\
        b = 1 \\
        c = 1
    \end{array}
\end{equation*}

\noindent Специфичното решение и множеството от решения на асоциираната хомогенната система са:
\begin{equation*}
    \{
        \left(\begin{array}{ c } 1 \\ 1 \\ 1 \end{array}\right)
    \}
    \quad
    \text{and}
    \quad
    \{
        \left(\begin{array}{ c } 0 \\ 0 \\ 0 \end{array}\right)
    \}
\end{equation*}


\subsubsection{Задача 3.15 e)}

\begin{equation*}
    \sysdelim..\systeme[x,y,z,w]{
        x + 2y - z = 3@p_{*},
        2x + y + w = 4,
        x - y + z + w = 1
    }
\end{equation*}

\noindent Решение:
\begin{equation*}
    \left(
        \begin{array}{ cccc|c }
            1 &  2 & -1 & 0 & 3 \\
            2 &  1 &  0 & 1 & 4 \\
            1 & -1 &  1 & 1 & 1
        \end{array}
    \right)
    \quad
    \overset{-2p_{1}+p_{2}}{\underset{-p_{1}+p_{3}}{\longrightarrow}}
    \quad
    \left(
        \begin{array}{ cccc|c }
            1 &  2 & -1 & 0 & 3 \\
            0 & -3 &  2 & 1 & -2 \\
            0 & -3 &  2 & 1 & -2
        \end{array}
    \right)
    \quad
    \overset{-p_{2}+p_{3}}{\longrightarrow}
    \quad
    \left(
        \begin{array}{ cccc|c }
            1 &  2 & -1 & 0 & 3 \\
            0 & -3 &  2 & 1 & -2 \\
            0 &  0 &  0 & 0 & 0
        \end{array}
    \right)
    \quad
    \longrightarrow
    \quad
\end{equation*}
\begin{equation*}
    \quad
    \longrightarrow
    \quad
    x = \frac{5 - z + 2w}{3}\ ,\quad
    y = \frac{2 + 2z + w}{3}\ ,\quad
    z = z\ ,\quad
    w = w
\end{equation*}

\noindent Множеството от решения е:
\begin{equation*}
    \{
        \left(\begin{array}{ c } 5/3 \\ 2/3 \\ 0 \\ 0 \end{array}\right) +
        \left(\begin{array}{ c } -1/3 \\ 2/3 \\ 1 \\ 0 \end{array}\right) z +
        \left(\begin{array}{ c } 2/3 \\ 1/3 \\ 0 \\ 1 \end{array}\right) w
        \ |\ z, w \in \mathbb{R}
    \}
\end{equation*}

\noindent Специфичното решение и множеството от решения на асоциираната хомогенната система са:
\begin{equation*}
    \{
        \left(\begin{array}{ c } 5/3 \\ 2/3 \\ 0 \\ 0 \end{array}\right) +
    \}
    \quad
    \text{and}
    \quad
    \{
        \left(\begin{array}{ c } -1/3 \\ 2/3 \\ 1 \\ 0 \end{array}\right) z +
        \left(\begin{array}{ c } 2/3 \\ 1/3 \\ 0 \\ 1 \end{array}\right) w
        \ |\ z, w \in \mathbb{R}
    \}
\end{equation*}

\subsubsection{Задача 3.15 f)}

\begin{equation*}
    \sysdelim..\systeme[x,y,z,w]{
        x + z + w = 4@p_{*},
        2x + y - w = 2,
        3x + y + z = 7
    }
\end{equation*}

\noindent Решение:
\begin{equation*}
    \left(
        \begin{array}{ cccc|c }
            1 & 0 & 1 &  1 & 4 \\
            2 & 1 & 0 & -1 & 2 \\
            3 & 1 & 1 &  0 & 7
        \end{array}
    \right)
    \quad
    \overset{-2p_{1}+p_{2}}{\underset{-3p_{1}+p_{3}}{\longrightarrow}}
    \quad
    \left(
        \begin{array}{ cccc|c }
            1 & 0 & 1  &  1 & 4 \\
            0 & 1 & -2 & -3 & -6 \\
            0 & 1 & -2 & -3 & -5
        \end{array}
    \right)
    \quad
    \overset{-p_{2}+p_{3}}{\longrightarrow}
    \quad
    \left(
        \begin{array}{ cccc|c }
            1 & 0 & 1  &  1 & 4 \\
            0 & 1 & -2 & -3 & 0 \\
            0 & 0 &  0 &  0 & -1
        \end{array}
    \right)
    \quad
    \longrightarrow
    \quad
\end{equation*}

\noindent Множеството от решения е:
\begin{equation*}
    \emptyset \text{ empty set}
\end{equation*}

\noindent Ако довършим решението с асоциираната хомогенната система:
\begin{equation*}
    \left(
        \begin{array}{ cccc|c }
            1 & 0 & 1  &  1 & 0 \\
            0 & 1 & -2 & -3 & 0 \\
            0 & 0 &  0 &  0 & 0
        \end{array}
    \right)
    \quad
    \longrightarrow
    \quad
    \begin{array}{c}
        x = 3 - z - w \\
        y = 2z + 3w \\
        z = z \\
        w = w
    \end{array}
\end{equation*}

\noindent Множеството от решения е:
\begin{equation*}
    \{
        \left(\begin{array}{ c } -1 \\ 2 \\ 1 \\ 0 \end{array}\right) z +
        \left(\begin{array}{ c } -1 \\ 3 \\ 0 \\ 1 \end{array}\right) w
        \ |\ z, w \in \mathbb{R}
    \}
\end{equation*}

\end{document}